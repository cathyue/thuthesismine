\chapter{结论}
\label{sec:conclu}

中心对称材料或非晶态材料的表面二次非线性已经在生物化学传感中得到了广泛的应用,而利用微型谐振腔对于光场的增强作用,可以极大地增强这种二次非线性。但就我们所知,几乎没有人对这些缺乏电偶极二次非线性的材料所制备的微腔进行深入的二次非线性研究。

因此,本文从理论和实验两方面入手。在理论上,首先探究了中心对称材料或非晶态材料中二次非线性的不同来源;接着探究了在微型谐振腔的框架中,这些二次非线性如何转化为基波和二次谐波之间的耦合系数;为了实际实验考虑,我们研究了有效产生二次谐波的前提,双共振条件,并提出了利用腔中显著的三次非线性效应(热效应和Kerr效应),来辅助调节腔模的位置,进而实现双共振条件;最后,我们分析了不同偏振在两种不同的非线性来源之下,产生二次谐波的效率。

实验上,设计了包含信号光光纤锥的实验装置,为能够有效收集二次谐波并观测其性质做好了铺垫;经过多次尝试,制备出了高品质因子,大小合适的微球腔,以及能够在两个波段同时达到临界耦合的双光纤锥,并探究出了灵活操纵双光纤锥的实验方法;在实验中证实了信号光光纤锥收集效率可以达到泵浦光光纤锥最优收集效率的13.73倍以上;探究了二次谐波信号功率随着泵浦光频率的变化关系,与理论中的近似洛仑兹线形有较好的对照;探究了二次谐波信号功率随泵浦光功率的变化关系,也与理论中的特征功率相对应;还给出了二次谐波产生功率和次数与偏振相对应的统计直方图,在实验上印证了理论中关于两种偏振二次非线性系数大小的相关结论;最后,在实验中还观测到了泵浦光和它对应的拉曼光子产生的二次合频光信号,拓宽了二次谐波波段的出射光谱范围。

本文填补了光学微腔非线性研究领域中的一大空白,拓宽了光学微腔出射光谱范围。非常重要的是,实验中在泵浦光功率882$\mu$W时(连续波激光器)就能够观测到功率约为0.4nW的二次谐波,而之前的实验中观测到二次谐波时,泵浦光功率高达209mW\cite{asano2016visible}。本实验展示了低功率二氧化硅微腔二次谐波产生的能力,为低功率生物化学分子的二次非线性方法传感作出了铺垫。同时,使用连续波激光器,与\cite{dominguez2011whispering}中使用脉冲激光器并将脉冲拉宽的实验方法相比,极大地简化了实验装置和操作。实验中的双光纤锥实验装置,作为良好的二次谐波收集系统,也可以应用在其他光学微腔弱信号探测和收集中。

由于双共振条件需要一定的腔内能量才能达到,而腔内能量又与输入光的频率密切相关,所以才得出类似于阶跃函数式的二次谐波功率-泵浦光功率依赖曲线,而非正规的二次曲线。为了得到二次曲线,应当有独立于泵浦光的腔内能量调控方式,并且有较好的腔内泵浦光能量衡量方法,这可以成为进一步实验的方向。