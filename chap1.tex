\chapter{引言}
\label{cha:intro}

\section{非线性光学}

非线性光学是光学中的一个重要分支,它是用来描述光在非线性介质中行为的一门科学。其中最为核心的问题是,电介质极化P随着电场E呈现怎样的非线性关系。通常情况下,在光强较弱时,大部分介质可以看作线性介质,而它们的非线性效应在光强很强的时候才能够体现出来。当光强极强的时候,甚至真空本身都可以被看作是非线性介质。线性光学当中非常重要的叠加原理到了非线性介质中都将不再成立。也正是因为探究非线性光学需要非常强的光强,几乎所有非线性光学的发展都在Maiman制造出第一台红宝石激光器之后才开始\cite{maiman1960stimulated}。1961年Peter Franken等人首次发现的二次谐波被视作非线性光学学科的开端\cite{franken1961generation}。从那之后,更多的非线性光学现象被发现并探究,包括二次谐波、三次谐波产生,差频产生,光学参量放大,光学参量震荡,光学参量产生,自发参量下转换,光学整流,非线性光-物质相互作用等等。这些基础非线性光学现象本身又带来很多有趣的光学效应,如自聚焦效应,自相位调制,交叉相位调制,光学孤子,四波混频,多光子吸收等等。还有一些非线性现象,是由外加电压、声波或材料中的声子引起的,如拉曼散射、布里源散射、电光效应和声光效应等(对上述非线性现象的介绍见\cite{boyd2003nonlinear})。

非线性光学的出现和发展极大地推进了光学和其他科学发展的疆域。非线性光学最常见的应用之一是频率转换,由于激光器输出波长与材料能级密切相关,所以可供选择的频段比较有限,但通过非线性材料,可以实现频率加倍,如,将Nd:YAG激光器输出的1064nm光倍频成为532nm的绿光,就产生了这个频段的激光器。另外,非线性效应应用在成像领域,还可以实现可逆光束用来进行图像重建。另外,在量子光学当中,双光子、多光子过程也扮演着非常重要的角色\cite{scully1999quantum}。

\section{微型谐振腔}

\begin{figure}
\centering
\includegraphics[width=14cm]{Microcavity.png}
\caption{不同种类的光学微型谐振腔,上排为高品质因子微腔,下排为超高品质因子微腔,左边一列为法布里珀罗腔,中间为回音壁微腔,右侧为光子晶体微腔。图引自\cite{vahala2003optical}}
\label{pic:Microcavity}
\end{figure}

微型谐振腔是一种能够将电磁场或光场局域在一个很小的范围内的结构。在空间上,这种结构的模式体积非常小。而在时间上,微型谐振腔非常高的光学品质因子($Q_o$)能够保证腔中的光子泄露出腔的速率非常慢。结合了小的模式体积和很高的品质因子,微型谐振腔能够显著地增强腔中光场的能量,比自由空间中的光强增加了几个数量级。因此,输入光功率仅有1mW时,腔中的能量密度就可以高达1GW/cm$^2$\cite{vahala2003optical}。这也使得微型谐振腔成为了一种用来探究非线性光学现象的极好的平台,如图\ref{pic:Microcavity}为几种典型的光学微腔示意图。用来制作微型谐振腔的材料大多数是非线性较弱的材料,如二氧化硅、氮化硅、碱金属氟化物等,对于这些材料来说,由于微型谐振腔本身对于光强极大的增强作用,使得在正常输入光下也可以看到在极高功率下才能观察到的非线性效应。而对于用具有强非线性材料制备的微型谐振腔,如铌酸锂微腔,在这种结构中可以显著降低某些非线性过程的阈值,并有效提高产率。在过去的十几年间,许多非线性现象相继在微型谐振腔中被观察到,譬如,拉曼散射和拉曼激光\cite{spillane2002ultralow, cai2000fiber, kippenberg2004ultralow},布里源散射和激光\cite{li2012characterization, li2013microwave, li2014low, loh2015dual},三次谐波产生\cite{carmon2007visible, farnesi2014optical}	和四波混频\cite{kippenberg2004kerr}等等。这些非线性现象不仅在物理上非常有趣,而且也有许多有价值的应用,譬如,利用拉曼激光可以实现高灵敏度生物传感\cite{ozdemir2014highly},利用布里源激光可以制备基于微腔的光学陀螺仪\cite{li2015microresonator}。

\section{二次谐波产生}

二阶非线性过程在量子光学中有许多有价值的应用,比如,非经典态光的产生\cite{scully1999quantum}	和量子纠缠对的产生\cite{xu2008second}。而二次谐波产生,作为最基本的二次非线性过程,通常是研究材料二阶非线性的第一步。在二次谐波产生过程当中,两个泵浦光子湮灭并产生一个拥有两倍频率的新光子\cite{boyd2003nonlinear},在正过程当中需要满足能量和动量守恒,属于典型的参量过程,而同时满足这两个守恒,也是有效产生二次谐波的条件之一,其中动量守恒又经常被称为相位匹配过程。

	二次非线性过程在非中心对称材料当中是最显著的非线性过程,这些材料包括铌酸锂、KTP等等。为了能够利用微腔增强作用,同时应用这些材料较大的二阶非线性系数,晶态铌酸锂被打磨成为微型谐振腔,$Q_o$值接近100亿\cite{ilchenko2004nonlinear}。与此同时,为了进一步能够提高转化效率,人们采用了周期极化的方法,来实现类相位匹配。之后,人们又利用铌酸锂晶体的双折射,来实现自然相位匹配,从而使得转化效率在低输入功率下提高了两个数量级\cite{furst2010naturally}。虽然打磨的方法能够制备品质因子极高的微腔,但是这种方法与标准CMOS工艺不兼容,无法批量生产,所以人们也在研究片上铌酸锂微腔,并用它成功产生了二次谐波\cite{lin2015second}。其他的一些具有较大非线性系数的材料,如砷化镓,也被用于二次谐波产生,而这里的相位匹配则利用了砷化镓晶体本身对称性的特点\cite{kuo2014second}。
	
	在中心对称材料当中,电偶极子的二阶非线性相应是被禁止的\cite{boyd2003nonlinear}。然而,大部分和CMOS工艺相容的高品质因子、超高品质因子微型谐振腔基本都是由中心对称材料制成,如,二氧化硅\cite{armani2003ultra, kippenberg2003fabrication}	和氮化硅\cite{levy2011harmonic}。一个利用这种微腔实现二次非线性过程的方法是在这些微腔的表面镀一层非线性分子\cite{xu2008second},比如,晶体紫罗兰\cite{dominguez2011whispering}。为了实现相位匹配,除了调节微型谐振腔的大小来控制色散并且利用更高阶径向模式以外,镀上去的非线性分子还可以被刻蚀出图案,如,可以刻成条纹状,利用非线性强度的周期性变化实现类相位匹配\cite{dominguez2011whispering}。然而,镀分子的过程本身非常复杂,而且会使得品质因子降低一到两个数量级\cite{xu2008second},这会使得非线性过程效率降低。
	
中心对称材料的一些本征性质也可以被用来产生二次谐波。在材料的表面或者两种材料的交界面上,原本的中心对称性被打破,进而可以引入二次非线性\cite{heinz1991second}。这种非线性过程对材料表面的性质非常敏感,譬如,粘附分子或者表面化学键的一些活动。这就使得这种二次谐波产生或二次合频过程可以作为非常有效的表面性质探测器\cite{shen1989surface, shank1983femtosecond, heinz1985study, tom1986investigation}。除了表面非线性之外,中心对称性并没有禁止电四极子和磁偶极子对于二次非线性过程的相应\cite{heinz1991second},它们也是引入二阶非线性的重要渠道之一。尽管在小电介质球集合中的二次谐波产生早在二十年前就被观察到了\cite{martorell1997scattering, maymo2006visible, shan2006experimental},但直到最近,利用中心对称材料制成的微型谐振腔中的二次谐波产生过程才首次被观察到\cite{asano2016visible}。
	
\cite{asano2016visible}中的工作仅给出了780nm附近的光谱图,没有给出泵浦光光谱,没有换算出二次谐波产生功率,对于二次谐波的来源,及二次谐波功率对泵浦功率、偏振等参数的依赖关系也没有深入地研究。本文将从理论和实验两方面对二氧化硅微腔中的二次非线性现象进行深入研究,对上述几个问题都将给出答案。第\ref{sec:WGM}章作为后文讨论的基础,将对回音壁微腔中的模式进行简介;第\ref{sec:nonlinearTheo}章将介绍二次非线性的理论,是分析后文的实验结果的基础;第\ref{sec:fab}章将简述实验设计和器件制备的过程;第\ref{sec:measurement}章分析实验结果;第\ref{sec:conclu}章对全文做出总结。
