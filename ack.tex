% 如果使用声明扫描页,将可选参数指定为扫描后的 PDF 文件名,例如:
% \begin{acknowledgement}[scan-statement.pdf]
\begin{acknowledgement}
首先要感谢我的毕业设计导师刘玉玺教授,他在量子信息学引论这门课上为我打开了新方向的大门。虽然我加入组里的时间并不长,但刘老师从毕业设计选题,相关参考文献等各个方面对我给予了非常多的帮助和指导。不论是单独讨论还是组会上的工作讲解,刘老师的意见都让我收获良多。刘老师自由宽容的风格让我尤其受益匪浅,学生有自己的想法,刘老师就会鼓励我们去究下去,本课题能够最终顺利完成,与刘老师给予我的鼓励和自由是分不开的。

接下来要感谢北京大学物理学院的肖云峰研究员,本课题最初就是在与肖老师的讨论中产生的,在整个课题的完成过程中,肖老师提供了很多大方向上非常有益的指导意见和许多值得探索的问题。肖老师的实验室具有非常多光学微腔的实验经验,他允许我在那里完成我毕业设计的实验部分,也极大地加快了我的实验进度。

在完成毕业设计的过程中,曹启韬师兄和彭湃同学为我介绍了光学微腔的基础知识,在他们的指引下,我开始了这方面的研究;实验方面,尤其要感谢曹启韬师兄,是他手把手地教我微球和光纤锥的制备方法,双光纤的实验设计也是从他的实验中得到的灵感。俞骁翀师兄非常乐于助人,在实验和仿真当中遇到问题都非常热心地帮我解决。实验室的其他师兄师姐,如王栗、张树昕、郅燕燕、唐水晶、徐达也给予了我很多帮助,在此一并表示感谢。

感谢加州理工学院的Kerry Vahala教授,易煦和杨起帆师兄。在他们组里进行的暑期科研中,我参与了微腔领域最前沿的科研,第一次接触到了光学微腔的实验,这让我对光学微腔研究有了更高层次的认识。他们对科研认真严谨的态度,探求问题极限的精神,和良好的科研直觉让我感受到了世界一流实验组的风格,从他们思考问题和待人处事的方式中,我都学到了很多,对我之后的科研,尤其是毕业设计有很大的帮助。易煦和杨起帆师兄还就本课题给出了一些建议,对解决课题中的问题起到了非常积极的作用。

还要感谢斯坦福大学的Tony Heinz教授,作为表面二次非线性领域的鼻祖,他提示我不可以忽视体多极引入的非线性,事实证明,该作用确实应当考虑进来。感谢哈佛大学的Marko Lo$\mathrm{\check{c}}$ar教授,对实验中证实二次谐波存在给出了建议。

感谢我的父母,无论我做出什么决定,包括选择现在所热爱的科研,他们都支持着我。感谢我的朋友们,在科研之余给予我的陪伴和鼓励,不论高潮或低谷,都让我更有动力地前行。感谢王之鑫学长,在科研方法和方向选择上给了我很多建议,我也一直以他为榜样,努力达到他的高度。

感谢清华大学给予我的平台和机会,感谢培养过我的老师们,感谢大学四年在各个方面帮助和鼓励过我的人,因为你们,才有我今天的一切。
\end{acknowledgement}
