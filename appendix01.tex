\chapter{外文资料的调研阅读报告}
\label{cha:engorg}

\title{Nonlinear Optics in Microresonators: Second Harmonic Generation and Solitons}


Optical microresonators are structures to confine light. Spatially, the light is confined in a small volume. Temporally, the high optical quality factor (Q) of microresonators makes sure the photons escape from the resonator in a low rate. Combining the small mode volume and high Q, the microresonator is able to enhance the intensity of light for orders of magnitude. Therefore, the intra-cavity intensity can reach 1GW/cm$^2$ with only 1mW of input power \cite{vahala2003optical, cai2000fiber, kippenberg2004ultralow}, which makes the microresonator a very good platform for the exploration of nonlinear optical phenomena, e.g. Raman scattering and lasing\cite{spillane2002ultralow, cai2000fiber, kippenberg2004ultralow}, Brillouin scattering and lasing \cite{li2012characterization, li2013microwave, li2014low, loh2015dual}, third harmonic generation \cite{carmon2007visible, farnesi2014optical} and four wave mixing \cite{kippenberg2004kerr}. These nonlinear phenomena are physically interesting and may also induce some applications, such as biosensing \cite{ozdemir2014highly} and optical gyroscope \cite{li2015microresonator}. In this review, we mainly focus on two nonlinear processes in whispering gallery microresonators, the second harmonic generation and the soliton, which are novel phenomena in silica microcavities. 

\section{Second harmonic generation}
Second order nonlinear processes are useful in many quantum optical applications, e.g. nonclassical light generation \cite{scully1999quantum} and quantum entanglement \cite{xu2008second}. Second harmonic generation, which often serves as the first step to explore the second order nonlinearity, is a fundamental nonlinear process in which two photons annihilate and generate a photon with doubled frequency \cite{boyd2003nonlinear}. This phenomenon is manifest in non-centrosymmetric materials like lithium niobate (LiNbO$_3$). In order to take advantage of the cavity enhancement and also the large second order nonlinear coefficient of LiNbO$_3$, the crystalline material was polished into whispering gallery mode microresonators with Q near 10 billion \cite{ilchenko2004nonlinear}. To further increase the conversion efficiency of the process, periodic poling technique was also introduced to achieve quasi-phase-matching. Natural phase matching, which leverages the birefringence of the crystal, was used to boost the efficiency for two orders of magnitude at low pump power \cite{furst2010naturally}. The polished microresonators are not compatible with standard micro-fabrication processes. Therefore, much effort has been made into fabricating on-chip high-Q LiNbO$_3$ resonator and demonstrating second order nonlinear phenomena \cite{lin2015second}. Other materials, e.g. GaAs, were also employed to generate second harmonic signal with different types of phase matching techniques \cite{kuo2014second}. 

In centro-symmetric materials, the dipole response of second order nonlinearity is forbidden \cite{boyd2003nonlinear}. However, most CMOS compatible high-Q microresonators are made of centrosymmetric materials like silica \cite{armani2003ultra, kippenberg2003fabrication} and silicon nitride \cite{levy2011harmonic}. One way to realize second order nonlinear process is to coat the microresonator with nonlinear molecules \cite{xu2008second}, for example, crystal violet \cite{dominguez2011whispering}. Apart from adjusting the size of the resonator and employing higher order radial modes to achieve natural phase matching \cite{xu2008second}, the coated molecules can also be written in a periodic pattern to achieve quasi-phase-matching \cite{dominguez2011whispering}. However, the coating process is complicated and would cause the degradation of Q for one or two orders of magnitudes \cite{xu2008second}. 

The intrinsic properties of centrosymmetric materials were also exploited in second harmonic generation. The surfaces and interfaces of the materials break the local centro-symmetry and bring in second order nonlinearity \cite{heinz1991second}. This nonlinearity is sensitive to the surface conditions, e.g. the adsorption of molecules and the activity of chemical bonds at surfaces, which renders the second harmonic generation or sum frequency generation effective tools as surface probes \cite{shen1989surface, shank1983femtosecond, heinz1985study, tom1986investigation}. In addition to the surface nonlinearity, the electric quadruple and magnetic dipole also induces second order nonlinearity \cite{heinz1991second}. Although the second harmonic generation had been demonstrated with small dielectric spheres two decades ago \cite{martorell1997scattering, maymo2006visible, shan2006experimental}, the process was not observed in microresonators made of centrosymmetric materials until recently \cite{asano2016visible}.

\section{Solitons}
First observed in a canal by Scott Russell \cite{russell1844report}, a heap of water propagating undistorted over a few miles got the name solitary wave or soliton for its shape and exotic behavior \cite{zabusky1965interaction}. Similar phenomena in optics have drawn great attention since the 1960s and have been studied extensively especially in the context of optical fibers \cite{agrawal2007nonlinear, hasegawa1973transmission, andrekson1999soliton}. Dispersion and nonlinear effects balance with each other to maintain the pulse width of solitons \cite{agrawal2007nonlinear, akhmediev2008three}. Loss and gain must also be balanced to keep the amplitude approaching constant considering the case of dissipative solitons \cite{akhmediev2008three}. Specifically in fiber cavities or microresonators, it has been demonstrated experimentally that a continuous wave pump laser can serve as the gain against the cavity loss while Kerr nonlinearity functions against anomalous dispersion. Therefore temporal solitons can exist in the cavities indefinitely \cite{leo2010temporal, herr2014temporal, yi2015soliton}. Compared with solitons in fiber cavities, the ones in integrable microresonators are converted directly from the continuous wave without external writing pulses, thus rendering them more fascinating \cite{herr2014temporal, yi2015soliton}. By controlling \cite{herr2014temporal} or even locking \cite{yi2015soliton} the detuning between the pump laser and the microresonator, one can make the resonator work in the single-soliton zone. A back tuning technique was also invented to achieve the single-soliton regime deterministically \cite{guo2016universal}. Besides the stable soliton regime, the breather soliton regime, in which the amplitude and width of soliton oscillate periodically with time, was also explored \cite{lucas2016breathing}. Combining with other nonlinear effects and various dispersion profiles, many interesting phenomena have been demonstrate, e.g. the Raman self-frequency shift \cite{karpov2016raman, yi2016theory}, Stokes soliton \cite{yang2016stokes} and dispersive waves \cite{brasch2016photonic, yang2016spatial, lucas2016study, yi2017single}. 

In the frequency domain, the solitons in microresonators correspond to frequency combs with well-defined envelopes. Compared with conventional laser frequency combs, the microcombs largely increase the repetition rate, which is important in applications like real-time spectroscopy \cite{bernhardt2010cavity} and astronomical spectroscopy \cite{steinmetz2008laser}. Additionally, the small foot print may also revolutionize the applications in instrumentation, time keeping, spectroscopy and a multitude of other areas with potential opportunities\cite{yi2015soliton}. Recently, the microcombs have already been employed in optical communications\cite{marin2016microresonator} and spectroscopy \cite{suh2016microresonator, liu2016frequency}.




