\chapter{实验设计与器件制备}
\label{sec:fab}
\section{实验设计}
\label{sec:ExpSetup}

从原理上考虑实验设计非常简单,只需要将泵浦光有效地耦合进腔,再通过合适的方式探测产生的二次谐波即可,不论是表面还是体多极二次非线性,都是腔内在的性质,不直接依赖于输入输出关系。\cite{carmon2007visible}中的三次谐波产生就只用了最简单的光纤锥耦合的方法,并用CCD摄像头收集产生的绿光。

对于本实验来说,由于信号光的光强非常弱,需要解决两个技术问题,一是如何有效收集二次谐波信号,二是如何有效探测二次谐波信号。如果采用\cite{carmon2007visible}中的方法,用传统CCD完成收集和探测的任务会导致实验装置复杂,且收集效率低。而如果仅用针对泵浦光的光纤锥收集,则由于泵浦光光纤锥是专门针对泵浦光设计的,仅在泵浦光波段能够达到临界耦合,在信号光波段由于无法较好地匹配光纤锥和球腔当中的波矢,收集效率非常低,另一方面,信号光在通信波段光纤里传播的过程中也会有较大的损失,这是三次谐波实验当中没有用光纤锥进行收集的主要原因。而我们认为,这也是之前实验当中几乎没有人观测到二次谐波信号的重要原因。

本实验中为了解决第一个问题,在泵浦光光纤锥之外,又专门加入了一根信号光光纤锥。在制备的时候,信号光光纤锥就设计为达到了信号光波段的单模条件,可以实现该波段的波矢与微球腔当中的波矢匹配,从而达到较高的耦合效率。而为了解决第二个问题,从信号光光纤锥中收集到的信号最终被送到EMCCD(电子放大电荷耦合器件)当中,与普通的CCD相比,EMCCD加入了电子放大功能,探测器部分也有专门的冷却装置,整体能够实现更高的信噪比,从而完成对弱信号的探测。

%如果仅使用这种实验设计,就完全与\ref{sec:2Resonance}中的理论分析一致,这将会导致二次谐波输出功率随着泵浦光功率呈一个阶跃式的变化,无法验证二次谐波功率与输入功率呈平方关系的特性。因而,在基础实验设计之上,再使用一路控制光源,专门用于调控腔的热效应及Kerr效应,使得达到双共振条件时的腔内泵浦光光强可以在较大范围内变化。

\begin{figure}
\centering
\includegraphics[width=13cm ]{ExpSetup.png}
\caption{实验设计示意图。pump laser:泵浦光激光光源;PC:偏振控制器;CPL:光耦合器;PM:光功率计;Microsphere:微球谐振腔;PD:光探测器;OSC:示波器;OSA:光谱仪;EMCCD:电子放大电荷耦合元件;Pump fiber taper:泵浦光波段相位匹配的光纤锥;Signal fiber taper:信号光波段相位匹配的光纤锥。}
\label{pic:ExpSetup}
\end{figure}

如图\ref{pic:ExpSetup}所示,泵浦光在1555nm附近,经过PC(偏振控制器)调节偏振后,输入泵浦光耦合光纤锥中。首先通过CPL(光纤分束器)分出一部分光进入OSA(光谱仪)中检测通讯波段光谱和拉曼等信号,另一部分光输入PD(光探测器)将光信号转化为电信号最后输入OSC(示波器)进行,监测泵浦光的透射谱。

二次谐波这一路光信号由二次谐波波段的光纤锥进行收集,并送入EMCCD(电子放大电荷耦合器件)中进行光谱探测。

\section{微球腔的制备}

%装置照片,制备步骤,成品照片,Q值,Transmission谱,半径,选取半径值的原因
\begin{figure}
\centering
\includegraphics[width=14cm ]{Setup_makingSphere}
\caption{微球腔制备实验装置图}
\label{pic:Setup_makingSphere}
\end{figure}

微球腔的制备主要使用图\ref{pic:Setup_makingSphere}中的装置,图片右侧为一台二氧化碳激光器,输出约10$\mu m$波长的激光,主要产生热效应,激光通过前方放置的透镜聚焦。制备时首先取一小段清洁的去包层光纤,下坠一个重物(如垫片),固定在玻片上,将玻片夹持在三维平移台上,通过图中所示的成像系统进行成像,用来调整光纤的位置,使得操作点在三个维度上都与激光器输出光的焦点重合。

通过函数信号发生器产生的方波来控制二氧化碳激光器的出光,方波为最大值时激光器出光,为零时不出光,这样可以通过控制方波的占空比来控制激光器的有效出光功率。

根据有效出光功率的不同,微球腔的制备可以分为三个阶段:第一阶段,采用较小的占空比(本实验中为7.5\%)出光约0.2s,配合重物,将光纤中的一段拉细拉长而不至于断裂,可以将原直径为125$\mu m$的光纤拉细为直径$10\mu m$以下,这是为了使得一般大小的微球(如,本实验中的62$\mu m$直径微球)的根部不会对整个微球的形状产生太大影响;第二阶段,将光纤上移至细光纤底部与粗光纤交界处在焦点上,提高占空比(本实验中为20\%),可以在很短时间内将光纤打断,下端随重物落下,上端则熔成一个近似于微球状的核;第三阶段,采用较高的占空比(本实验中为17\%),聚焦在微球下方约200$\mu m$处,对微球和光纤进行加热,使得微球熔化上卷,不断消耗细光纤,微球的体积也逐渐增大,当微球上卷到细光纤部分只剩下约10$\mu m$的时候,停止加热,微球腔制备完成。

在实验中,可以通过经验性地控制第二阶段打断细光纤的位置来控制球腔的大小,但做不到非常精确。因此也没有办法通过这种方法精确控制腔的色散,所以需要用\ref{sec:2Resonance}节中的方法来实现双共振。理论仿真的结果表明,在微球直径为60$\mu m$或略大一些时,能够在1550nm波段比较接近双共振条件,之后再进用三次非线性辅助的方法接近双共振条件会更容易。因此在实验中,我们尽可能地选用这样尺寸的微球,最终测得下一章中用来产生二次谐波的微球直径为62$\mu m$。


\section{双光纤的制备与操控}

%拉光纤装置照片,步骤,成品照片,透过率,操控光纤耦合方法,耦合球腔照片

\begin{figure}
\centering
\includegraphics[width=14cm ]{Setup_pullingFiber}
\caption{双光纤制备实验装置图}
\label{pic:Setup_pullingFiber}
\end{figure}

图\ref{pic:Setup_pullingFiber}所示是用来制备双光纤的实验装置。从原理上来讲,光纤锥是通过氢氧焰对光纤中部进行加热,配合两边电机驱动将光纤拉长,使得受热部分变细,直到受热部分的直径达到该波段的单模条件,此时光纤锥中的波矢与球腔中基模的波矢相近,能够实现较好的耦合效率。

在实验上,可以通过调节火焰的位置找到一个最佳加热点,使得在达到单模条件时由于光纤变细造成的损耗较小,针对1550nm波段的光纤锥,在最优位置,损耗可以降低到1$\%$以下。而对于双光纤来说,主要有两个难点,一是使得两根光纤在各自波段能几乎同时达到单模条件,不至于使得后达到单模条件的光纤制备好时第一根光纤已经变得太细,影响耦合;另一个难点是要同时使得两根光纤的损耗尽可能小。经过反复试验,得到一组参数,可以稳定地使两根光纤同时达到单模条件。对于这组参数,泵浦光光纤锥的损耗为3$\%$,信号光光纤锥的损耗为35$\%$。

具体制备步骤如下:首先将两根光纤分别通1555nm波段和778nm波段的光,接在光探测器上随时检测功率;之后在两根光纤中部选取合适的位置各自拨开长约1cm的包层,用酒精擦拭干净,卡在最细的夹槽中,需要注意的是,最细的夹槽是为单光纤设计的,两根光纤一起放有时会在拉动过程中打滑,不利于制备,所以实验中用贴纸做固定;接下来点燃氢氧焰,对光纤中部预热3分钟;预热好后,控制电机向两侧拉光纤,同时通过光探测器检测两根光纤的输出功率,在达到单模条件之前,光纤输出功率会反复震荡,最终在达到单模条件时稳定,此时同时关闭火焰和电机;最后利用成像系统检查两根光纤是否完好,是否过于松弛影响耦合等,并进行相应调整。

\begin{figure}
\centering
\includegraphics[width=16cm ]{DualFiber}
\caption{双光纤锥的顶视图,上面是泵浦光光纤锥,下面是信号光光纤锥}
\label{pic:DualFiber}
\end{figure}

图\ref{pic:DualFiber}是双光纤的顶视图,上面的是泵浦光光纤锥,下面是信号光光纤锥,可以看到泵浦光光纤锥比信号光光纤锥要粗,这是由于泵浦光波长约为信号光的两倍,在达到单模条件时,光纤锥直径也会更大。

\begin{figure}
\centering
\includegraphics[width=16cm ]{Fiber_sphere}
\caption{进行耦合前的光纤锥和微球腔顶视图}
\label{pic:Fiber_sphere}
\end{figure}

图\ref{pic:Fiber_sphere}是进行耦合前的光纤锥和微球腔顶视图,微球腔的固定是将光纤杆与地面垂直地贴在垫块上,故在顶视图中看不到光纤杆。

\begin{figure}
\centering
\includegraphics[width=16cm ]{coupling}
\caption{耦合时的光纤锥和微球腔顶视图}
\label{pic:coupling}
\end{figure}

在图\ref{pic:Fiber_sphere}中可以看到,两根光纤间的距离(取决于未去包层的光纤直径)约为250$\mu m$,比微球腔直径大得多,无法进行直接耦合,在实验中,加了一个固定在三维纳米平移台上的尖硅片,用硅片挑住信号光光纤锥,可以调节它与微球腔的相对距离和高度,移到一个合适的位置进行耦合,如图\ref{pic:coupling}所示,左下方阴影为尖硅片。

在实验中,泵浦光光纤在找到了波矢匹配位置之后就保持不动,为了实现较好两个波段的较好耦合,首先将球腔靠近泵浦光光纤锥,通过调节球腔位置并检测投射谱,使其与泵浦光光纤实现临界耦合。调节好球腔位置之后,再通过硅片操纵信号光光纤锥使得它与球腔实现较好耦合,在这个过程中,信号光光纤中输入778nm附近的激光,通过监测投射谱来监测信号光耦合情况。在采集二次谐波数据时,为了保护EMCCD,输入光光源处于关闭状态。
