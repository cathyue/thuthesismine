\thusetup{
  %******************************
  % 注意:
  %   1. 配置里面不要出现空行
  %   2. 不需要的配置信息可以删除
  %******************************
  %
  %=====
  % 秘级
  %=====
  %secretlevel={秘密},
  %secretyear={10},
  %
  %=========
  % 中文信息
  %=========
  ctitle={二氧化硅微腔中的二次非线性},
  %cdegree={工学硕士},
  cdepartment={微电子与纳电子学系},
  cmajor={	微电子科学与工程},
  cauthor={张雪悦},
  csupervisor={刘玉玺教授},
  %cassosupervisor={陈文光教授}, % 副指导老师
  %ccosupervisor={某某某教授}, % 联合指导老师
  % 日期自动使用当前时间,若需指定按如下方式修改:
  % cdate={超新星纪元},
  %
  % 博士后专有部分
  %cfirstdiscipline={计算机科学与技术},
  %cseconddiscipline={系统结构},
  %postdoctordate={2009年7月——2011年7月},
  %id={编号}, % 可以留空: id={},
  %udc={UDC}, % 可以留空
  %catalognumber={分类号}, % 可以留空
  %
  %=========
  % 英文信息
  %=========
  %etitle={An Introduction to \LaTeX{} Thesis Template of Tsinghua University v\version},
  % 这块比较复杂,需要分情况讨论:
  % 1. 学术型硕士
  %    edegree:必须为Master of Arts或Master of Science(注意大小写)
  %             “哲学、文学、历史学、法学、教育学、艺术学门类,公共管理学科
  %              填写Master of Arts,其它填写Master of Science”
  %    emajor:“获得一级学科授权的学科填写一级学科名称,其它填写二级学科名称”
  % 2. 专业型硕士
  %    edegree:“填写专业学位英文名称全称”
  %    emajor:“工程硕士填写工程领域,其它专业学位不填写此项”
  % 3. 学术型博士
  %    edegree:Doctor of Philosophy(注意大小写)
  %    emajor:“获得一级学科授权的学科填写一级学科名称,其它填写二级学科名称”
  % 4. 专业型博士
  %    edegree:“填写专业学位英文名称全称”
  %    emajor:不填写此项
  %edegree={Doctor of Engineering},
  %emajor={Computer Science and Technology},
  %eauthor={Xue Ruini},
  %esupervisor={Professor Zheng Weimin},
  %eassosupervisor={Chen Wenguang},
  % 日期自动生成,若需指定按如下方式修改:
  % edate={December, 2005}
  %
  % 关键词用“英文逗号”分割
  %ckeywords={\TeX, \LaTeX, CJK, 模板, 论文},
  %ekeywords={\TeX, \LaTeX, CJK, template, thesis}
}

% 定义中英文摘要和关键字
\begin{cabstract}
表面对称性破缺带来的二次非线性效应已经被广泛应用于生物、化学传感当中,由于光学方法非接触、速度快、灵敏度高等特点,十几年间该方法得到了迅速发展。中心对称性材料或非晶态材料的电偶极二次非线性缺失使得其表面二次非线性成为易于实验观测的非线性,这类材料从而成为二次非线性生化传感的主角。但表面二次非线性通常极弱,实验上只有使用 高功率激光器才能观测到显著现象。高品质因子的光学谐振腔具有显著增强光场的作用,从而增强非线性过程,降低所需要的泵浦光功率。

本毕业设计课题就以熔融二氧化硅微球腔中的二次谐波产生为核心,从理论上探究了该背景下二次非线性的来源,及使得二次谐波有效产生的双共振条件;实验上设计了双光纤锥装置来有效激发并收集二次谐波,并证实了其收集效率为传统微腔实验装置的13倍以上。实验中,仅用882微瓦的泵浦光输入功率便可以观察到二次谐波,比先前连续光激光器的相关工作降低了两个数量级以上,比脉冲激光器的相关工作降低了五个数量级,转化效率高达$4\times10^{-4}W^{-1}$。观测到二次谐波之后,我们深入探究了二次谐波功率随泵浦光频率、功率和偏振的变化关系,并从理论上给出了解释。

就我们所知,本课题第一次对中心对称材料回音壁模式微腔中的二次谐波产生做了系统性的研究,为低功率、连续光二次非线性生化传感提供了可能性。课题中的理论和实验装置也便与拓展到其他微腔非线性研究当中。


%  关键词是为了文献标引工作、用以表示全文主要内容信息的单词或术语。关键词不超过 5
%  个,每个关键词中间用分号分隔。(模板作者注:关键词分隔符不用考虑,模板会自动处
%  理。英文关键词同理。)
\end{cabstract}

% 如果习惯关键字跟在摘要文字后面,可以用直接命令来设置,如下:
 \ckeywords{光学微腔, 谐波产生,中心对称材料, 非线性}

\begin{eabstract}
Second order nonlinear effects induced by surface symmetry broken have sought many applications in biological or chemical sensing. It has the advantages of contact-free, short response time and high sensitivity, which contribute to its fast development in the past decades. In centro-symmetric or amorphous materials, the electrical dipole induced second order nonlinearity is forbidden, thus making the surface nonlinearity manifests itself in second order nonlinear effects. Therefore these materials have been used widely in the sensing measurements. The surface nonlinearity is intrinsically weak and high power lasers are required to make second harmonics observable. To reduce the pump power and possible damage to samples, microresonators can be incorporated as a platform to enhance the weak nonlinearity.

In the thesis, we focused on the second order nonlinear effects in a whispering-gallery microsphere made of fused silica. The origin of these effects and the phase-matching condition have been investigated theoretically. An experiment set-up with dual tapered fibers to effectively pump and collect the signals have been developed and the collecting efficiency was measured to be 13 times larger than the optimized traditional single tapered fiber set-up. We observed the second harmonics with a pump power of only 882$\mu$W, which is more than 2 orders of magnitude smaller than the previous continuous wave measurement and 5 orders of magnitude smaller than the pulsed laser measurements. We also looked into the dependence of the second harmonic power on the pump detuning, power and polarization, which agree with our theory.

To the best of our knowledge, it is the first time that the second order nonlinear effects have been systematically investigated. The thesis provides possibility to low power, continuous wave biological or chemical sensing and also opens up the second order nonlinearity research opportunities for other centro-symmetric materials.
\end{eabstract}

\ekeywords{Microresonators, Nonlinear wave mixing, Harmonic generation , Centro-symmetric materials}
