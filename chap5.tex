\chapter{实验结果及分析}
\label{sec:measurement}

\section{二次谐波信号的产生及两根光纤锥收集情况对比}

%SH spectrum, pump spectrum, Q, f1f2
\begin{figure}
\centering
\includegraphics[width=13cm ]{FigSHspectrum}
\caption{二次谐波产生光谱图。上图为二次谐波光谱,下图为泵浦光光谱,插图为二次谐波产生的能级示意图}
\label{pic:FigSHspectrum}
\end{figure}

使用\ref{sec:ExpSetup}节中的实验装置,在功率和偏振合适的情况下,可以观察到二次谐波信号,如图\ref{pic:FigSHspectrum}所示,此时的微腔处光纤中泵浦光功率为4.46mW。二次谐波功率的最大值出现在777.75$\pm0.34$nm处,泵浦光功率最大值出现在1555.14nm处,满足二倍频关系。


二次谐波信号与微球腔密切相关,如果将微球腔移开或将微球腔和信号光光纤锥在保持间距不变的情况下一起移开,信号光光纤锥中都不会收集到任何信号。

此外,二次谐波信号还与微球腔的模式密切相关。由于热效应和Kerr效应的影响,在波长逐渐增大耦合进腔膜中时,耦合会逐渐加深,且在较长一段波长范围之内都处在该模式之中,但一旦泵浦光追上腔膜之后,就会跳出该模式,此时反向调节泵浦光,不会立刻进入该模式\cite{carmon2004dynamical},因此可以利用此特性,使得一些波长点只在增加波长调节的过程中在该模式内,而反向调节的过程中,不在该模式内。图\ref{pic:FigSHspectrum}所示的波长满足这样的条件,出模式之后反向调节的过程中在该波长处依然没有进入模式,而在EMCCD上也完全观测不到该图所示的二次谐波信号。由此排除了该信号是光栅二级衍射信号的可能性,因为光栅二级衍射信号只与波长和泵浦光功率有关而与模式无关。

\begin{figure}
\centering
\includegraphics[width=12cm ]{Qfit_w}
\caption{产生二次谐波的模式透射谱。a, 信号光光纤靠近微球腔时的透射谱。b, 信号光光纤移开微球腔时的透射谱。}
\label{pic:Qfit_w}
\end{figure}

为了进一步观察该模式的特性,实验中降低功率到热效应可以忽略,能够看出模式的洛仑兹线形,此时的模式在约1555.02nm处。图\ref{pic:Qfit_w}a是信号光光纤在球腔附近时的模式透射谱,b是信号光光纤远离球腔时的模式透射谱。可以看到该模式存在模式劈裂,这是由于腔表面存在散射点,使得原本频率简并的正向传播和背向传播的行波模式通过散射点耦合起来,如果将模式正交化处理,则两个新的模式为驻波模式,它们与散射点的相对位置不一样,从而导致模式频率不再简并,而产生劈裂\cite{cao2017experimental},一般在品质因子较高的情况下容易观察到这样的现象。图中的蓝线是理论拟合结果,由该结果可以推知,信号光光纤锥接近球腔时的本征品质因子为3.62$\times 10^7$,而将信号光光纤锥移开时的本征品质因子为4.83$\times 10^7$,这里的本征品质因子是相对于泵浦光光纤锥来说的,即信号光光纤锥的影响也包含在里面。由于信号光光纤锥的存在,使得腔中光场多了一条耗散通道,所以本征品质因子有所降低。

为了证明信号光光纤锥虽然降低了品质因子,但能够提高收集到的二次谐波信号功率。实验中选取了另一个模式,在泵浦光输入功率约7.45mW时进行探测。用图\ref{pic:ExpSetup}所示的实验装置图,在信号光光纤锥中收集到二次谐波信号功率约为5nW。接着将信号光光纤锥移开,将泵浦光光纤锥出口直接接在EMCCD上,仅用泵浦光光纤锥完成信号的激发和收集。直接移开信号光光纤锥,泵浦光光纤锥中收集到的二次谐波信号不足0.1nW,通过纳米平移台精细调节球腔相对泵浦光光纤锥的位置,可以得到一个二次谐波功率的最大值,为0.364nW。可见,信号光光纤锥对于观察二次谐波,尤其是功率较弱的 二次谐波信号具有重要作用。

同样,只用泵浦光光纤锥也进行了正向、反向调节波长的测试,只有在模式中特定位置才能够看到二次谐波信号,如果不在模式中,即使有相同的泵浦光功率和波长,也看不到信号。进一步证明了该信号来自于球腔,而非EMCCD的光栅二级衍射。

\begin{figure}
\centering
\includegraphics[width=14cm ]{Figf1_f2}
\caption{两根光纤锥收集信号对比,红线为用信号光光纤锥收集到的信号,耦合位置未经过优化,黑线为将信号光光纤锥移开后,只用泵浦光光纤锥在产生二次谐波功率最大的耦合位置收集到的二次谐波信号。}
\label{pic:Figf1_f2}
\end{figure}

\section{二次谐波信号对失谐和输入光功率的依赖关系}

%two figures, movement of 780nm, 

\begin{figure}
\centering
\includegraphics[width=13cm ]{TransSHpower}
\caption{二次谐波信号与失谐的关系。a,归一化的泵浦模式透射谱(波长依次增大)和归一化的二次谐波信号功率随泵浦光波长的变化,理论计算。b,二次谐波功率较大的部分随失谐变化关系。}
\label{pic:TransSHpower}
\end{figure}

图\ref{pic:TransSHpower}展示了二次谐波信号随失谐的变化关系,失谐是泵浦光波长和冷腔模(输入光功率极小,热效应可忽略时的腔膜波长,在本实验中为1555.02nm)之差。a图是以实验中拟合得到的数据作为参数,使用\ref{sec:2Resonance}节中的理论得出的泵浦光腔模透射谱和二次谐波功率随泵浦光波长变化的曲线。透射谱由于热效应和Kerr效应,展宽成为一个三角形\cite{carmon2004dynamical},由于此时腔模处于欠耦合状态,故透射谱最低点不到零。根据双共振条件的理论,在泵浦光波长约1555.3nm附近泵浦光波长的一半与二次谐波模式达到共振,因而二次谐波模式产生了一个极大值。

b图展示了二次谐波功率极大值附近的情况,同样使用\ref{sec:2Resonance}节中的理论进行拟合,可以得出如下关系,
\begin{equation}
\frac{\kappa_{2e}+\kappa_{20}}{2} = (2-D_{12})\times 224.9\mathrm{ MHz}
\end{equation}

其中$D_{12}$是单位泵浦光频率变化带来的二次谐波模式频率移动。由于实验上无法确定二次谐波的具体模式,故无法进一步确认$\kappa_2$和$D_{12}$的值。

\begin{figure}
\centering
\includegraphics[width=14cm ]{FigSHpower-inputPw}
\caption{二次谐波信号与泵浦光功率的关系。蓝点为实验数据,红线为理论拟合曲线}
\label{pic:FigSHpower-inputPw}
\end{figure}

图\ref{pic:FigSHpower-inputPw}所示为二次谐波信号与泵浦光功率的关系,与图\ref{pic:P2change_ed2_ai}所示的理论预测非常接近。实验数据是固定一个泵浦光功率之后,向波长增大的方向扫描泵浦光波长,由上文分析的二次谐波功率随失谐的变化关系,二次谐波功率会逐渐增加,但如果功率较小,使得还未达到二次谐波功率较大的时候,泵浦光就已经追上腔模,之后会跳出腔模,则二次谐波功率极小。当泵浦光功率逐渐增大,直到能够覆盖使二次谐波功率较大的失谐时,二次谐波最大功率先是在即将失稳处(即泵浦光追上腔模)达到所有失谐中的最大值(固定泵浦光功率不变)。当泵浦光功率增大到能够取到的失谐范围覆盖了二次谐波功率极大值(即图\ref{pic:TransSHpower}b中的峰值),二次谐波功率的最大值每次都是在该峰值处读取,又由于不改变失谐而直接改变泵浦光功率对于腔内能量的增加作用极小,因而此后二次谐波功率极大值随着泵浦功率的增加变化不是很大。

\section{二次谐波信号对偏振的依赖关系}

%polarization figure, theories, 
\begin{figure}
\centering
\includegraphics[width=14cm ]{FigSHexamples}
\caption{不同模式产生的二次谐波信号,强度已经归一化。}
\label{pic:FigSHexamples}
\end{figure}

除了上面看到的两个模式产生二次谐波以外,微球腔中还有许多其他不同模式也可以产生二次谐波,实验中泵浦光波长从1545nm到1565nm区间内扫描(超过两个自由光谱范围),整个区间内均有模式能产生二次谐波,如图\ref{pic:FigSHexamples}所示,产生的二次谐波波长也覆盖了773nm至782nm的范围。

针对两个不同的模式比较它们的二次谐波强度没有直接意义,但由于可以观察到很多产生二次谐波的模式,可以研究二次谐波产生的统计特性,其中一个重要方面在于,对于偏振的依赖性。但在本实验中无法直接判别模式是TE还是TM偏振,只能标记为偏振1和偏振2,其中图\ref{pic:FigSHspectrum}所示的模式来自偏振1.实验中,以7.45mW的输入光功率分别采用两种偏振对1545nm到1565nm区间进行了三次扫描,记录下每个对应模式产生二次谐波的最大值,绘制成如图\ref{pic:PolarizationHistogram}所示的统计直方图。由该图可以看出,偏振1对应的模式能够产生更多功率更高的二次谐波,即两个偏振对于产生二次谐波来说并不是平等的。

\begin{figure}
\centering
\includegraphics[width=14cm ]{PolarizationHistogram}
\caption{不同偏振中模式产生二次谐波的统计直方图}
\label{pic:PolarizationHistogram}
\end{figure}

解释这个现象需要考虑二氧化硅微球腔中二次非线性的来源。首先考虑表面对称性破缺引起的二次非线性,在非晶态材料当中,表面二次非线性系数是一个张量,只有三个分量不为零,分别是$\chi_{\perp \perp \perp}$,$\chi_{\perp \parallel \parallel}$和$\chi_{\parallel \parallel \perp}$,对于二氧化硅来说,三个系数分别为59,3.8和7.9(单位均为$1\times 10^{-22} m^2/V$)。在微球腔中,$\chi_{\perp \perp \perp}$,主要对应着泵浦光和二次谐波均为TM模式,$\chi_{\perp \parallel \parallel}$对应着一个TE泵浦模式的光子和一个TM泵浦模式的光子结合产生了一个TE模式的二次谐波光子,一般这是一个二次合频过程而非二倍频过程,由于本实验只使用一台泵浦光激光器,故不研究此过程。$\chi_{\parallel \parallel \perp}$则对应着两个TE模式的泵浦光子结合产生一个TM模式的二次谐波光子。从系数大小上来看,TM泵浦产生二次谐波的效率应当最高。

另一个二次非线性的来源是体多极效应。由式\ref{eq:gb}可知,体非线性系数为一个标量,而积分中含有空间散度,故这个等效体二次非线性系数与模式的空间分布密切相关。对于TM模式来说,由于电场主要分布在$\mathbf{\hat{r}} $向,故微分也是在这个方向。$\mathbf{\hat{r}} $向本身不存在对称性,因而微分之后与二次谐波模式的电场及泵浦模式电场的乘积空间积分不会为零。TE模式的电场分布在$\mathbf{\hat{\theta}} $向,由于球对称性,这个方向本身也具有对称性,如果泵浦光选取TE角向基模,即,$m=l$,对于角向为偶函数,微分之后是奇函数,而根据\ref{sec:2Resonance}节中的积分不为零条件和$M\le L$,二次谐波模式也只能为TE角向基模,为角向偶函数,因而整体的交叠积分为零,即,TE角向基模不能通过体二次非线性产生二次谐波。泵浦光与二次谐波模式都为TE角向二阶模时,空间积分不为零,可以产生二次谐波。另外需要注意的是,无论哪种二次非线性来源,都需要通过径向高阶模式(在实验考虑的条件下为径向二阶模式)来达到初步双共振条件。图\ref{pic:TM1zoomwarrow_1_9319}所示为通过体二次非线性能够产生二次谐波的最低阶模式。

\begin{figure}
\centering
\includegraphics[width=14cm ]{TM1zoomwarrow_1_9319}
\caption{通过体二次非线性能够产生二次谐波的最低阶模式,电场强度切面分布仿真图。箭头为局部电场方向。a, 泵浦光TM基模,仿真波长1551.8nm。b, 二次谐波TM径向二阶角向基模$M=L, Q=2$,仿真波长775.96nm。c, 泵浦光TE角向二阶模$m=l-1, q=1$,仿真波长1549.2nm。d, 二次谐波TE径向二阶角向二阶模$M=L-1, Q=2$,仿真波长775.30nm。}
\label{pic:TM1zoomwarrow_1_9319}
\end{figure}

通过上述分析可知,TM模式在径向微分,且对模式没有附加要求,而TE模式在角向微分,且需要高阶模式。由图\ref{pic:TM1zoomwarrow_1_9319},由于模式在径向压缩得更严重,因而在同样归一化的条件下TM模式应当得到更大的微分值,且基模有利于达到更好的光纤锥耦合条件。为了定量说明这一点,使用式\ref{eq:gb}计算了直径62$\mu m$的微球腔中,如图\ref{pic:TM1zoomwarrow_1_9319}所示两组模式的二次谐波耦合系数,得到TM模式耦合系数的绝对值为TE模式耦合系数绝对值的18.62倍。

综合以上分析,虽然不同模式有所不同,但统计一个自由光谱范围中的全部模式,TM模式产生二次谐波的强度应当比TE模式要大,因此在实验上会观察到两个偏振产生二次谐波模式的数量和强度的统计特性不相同的情况。


\section{拉曼和二次合频信号}

%figures, a little bit about Raman

光子与二氧化硅中晶格振动产生的光学声子产生相互作用,会存在拉曼散射效应\cite{boyd2003nonlinear},产生一个波长更长的光子,拉曼光子的频率取决于材料本身的拉曼增益带。由于微腔对光场的增强作用,可以显著降低拉曼产生的阈值\cite{spillane2002ultralow, cai2000fiber, kippenberg2004ultralow}。 

在本次实验条件下,在某些模式中可以观察到拉曼信号,且在合适的条件下,拉曼光子与泵浦光子通过二次非线性效应进行合频,可以产生一个二次合频光子,进一步拓宽了可观测的频率范围。

\begin{figure}
\centering
\includegraphics[width=14cm ]{FigRamanSF}
\caption{二次合频信号。a, 二次合频信号光谱图,波长为804.67nm。b,泵浦光波段光谱图,泵浦波长1550.88nm,拉曼光波长1674.22nm,插图为光子转换示意图。}
\label{pic:FigRamanSF}
\end{figure}

如图\ref{pic:FigRamanSF}所示为二次合频信号光谱图及其对应的泵浦波段光谱图,分析上图可以得出波长关系,$1/1550.88nm+1/1674.22nm = 1/804.63nm \approx 1/804.67nm$,满足二次合频关系。